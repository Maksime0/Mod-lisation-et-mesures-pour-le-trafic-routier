\documentclass[12pt, openany]{report}
\usepackage[utf8]{inputenc}
\usepackage[T1]{fontenc}
\usepackage[a4paper,left=2cm,right=2cm,top=2cm,bottom=2cm]{geometry}
\usepackage[french]{babel}
\usepackage{setspace}
\usepackage{hyperref}
\usepackage{libertine}
\usepackage[pdftex]{graphicx}
\usepackage{color}
\usepackage{geometry}
\usepackage{amsmath}
\usepackage{multirow}
\usepackage[table,xcdraw]{xcolor}
\usepackage{amsmath}
\usepackage{algorithm}
\usepackage[noend]{algpseudocode}
\usepackage{enumitem}
\usepackage[export]{adjustbox}
\usepackage{lscape}
\usepackage{pgfgantt}
\usepackage{enumitem,amssymb}
\newlist{todolist}{itemize}{2}
\setlist[todolist]{label=$\square$}
\usepackage{pifont}
\usepackage{diagbox}
\usepackage{slashbox}
\usepackage[table]{xcolor}

\begin{document}

\centering \textbf{\Large Première expérience}

\flushleft
Il est compliqué de faire une mesure de la densité $\rho$ directement car nous devrions être capable de compter le nombre de véhicule qui se trouve sur un tronçon de longue distance à un instant $t$, ce qu'il se passe à un endroit $x$ appartenant au tronçon et peut être trop loin pour être vu.\\
Nous pouvons cependant mesurer le flux $f$ à un endroit $x$ et supposé qu'il est a peu près constant sur une partie du tronçon $(x-dx,x+dx)$ que nous appelerons $(E-S)$, cela correspond un un système ouvert un mécanique des fluides sur lequel nous pouvons appliquer le théorème de Bernoulli. C'est à dire que le flux entrant est égal au flux sortant et ce flux est le même sur une petite partie du tronçon ($E-S$).\\

\centering\begin{tikzpicture}
	\draw (0,0)node[left]{A} --(8,0) node[right]{B};
	\draw [line width=1pt] (5,0)node{\tiny$|$} node[below]{E} --(6,0) node{\tiny$|$} node[below]{S};
\end{tikzpicture}
\flushleft

Nous pouvons aussi mesurer la vitesse des véhicules et en déduire la moyenne $v$.\\
Nous pouvons donc récupérer la densité $\rho$ grâce à la formule $f=\rho v \Leftrightarrow \rho =\dfrac{f}{v}$\normalsize\\
Nous pouvons bien sur mesurer le flux $\rho$ en comptant le nombre de véhicule $N$ qui passe sur une durée $T$ qui doit être suffisamment longue pour une mesure précise mais pas trop longue car la densité $\rho$ dépend du temps.

$$ 
\left\{
    \begin{array}{l}
        f=\rho v \\
        f=\frac{N}{T}
    \end{array}
\right. \Rightarrow
\rho = \frac{N}{v.T}
$$

Après un ensemble de mesures, nous pourrons nous servir des couples de $(\rho , v)$ pour déterminer la droite passant au plus proche de chaque point en utilisant par exemple la méthode des moindres carrés.
Nous pourrons donc déterminer $\rho_{max}$ et $v_{max}$ grâce à :
$$ 
\left\{
    \begin{array}{lll}
        v(0)&=&v_{max}\\
        v(\rho_{max})&=&0
    \end{array}
\right.
$$

Nous pourrons aussi donné la courbe approchée de $f(\rho)$ qui vaut :
$$
\begin{array}{ll}
     & f(\rho) = \dfrac{v_{max}}{\rho_{max}}(\rho_{max}-\rho)\rho\\
     \Leftrightarrow & f(\rho)=v_{max}.\rho - \dfrac{v_{max}}{\rho_{max}}\rho^2
\end{array}
$$

On aura donc un flux maximal $f_{max} $ atteint en $\rho_c$ en $\rho_c=\dfrac{\rho_{max}}{2}$.
$$ f(\rho_c)=f(\dfrac{\rho_{max}}{2})=f_{max}$$

De cette expérience, nous pourrons observé dans un premier temps si le modèle LWR suit bien l'expérience mené, et ensuite un nuage de point correspondant à une fonction de la densité $\rho$ en fonction du temps $t$ : $\rho (t)$ sur le tronçon $(E-S)$ et ansi voir la forme de la fonction.


\end{document}